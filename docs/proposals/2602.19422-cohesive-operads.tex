% PlanetMath proposal draft for arXiv:2602.19422
% Source corpora: arXiv math.CT eprints, storage/mo-processed-gpu, storage/math-processed-gpu
\section*{Cohesive operads and shape theory}
\subsection*{Source}
\begin{itemize}
  \item arXiv:2602.19422, "Cohesive operads and shape theory" (2026-02-21)
  \item StackExchange cross-links via storage/mo-processed-gpu + math-processed-gpu
\end{itemize}
\subsection*{Synopsis}
Cohesive operads live inside cohesive $\infty$-topoi and encode shape-theoretic invariants. The authors:
\begin{enumerate}
  \item define operads internal to a cohesive topos and explain how spans realize operations;
  \item prove a shape-invariance theorem comparing cohesive operads with classical Borsuk shape;
  \item relate the construction to condensed symmetry actions; and
  \item provide test cases based on condensed tori.
\end{enumerate}
\subsection*{MathOverflow cues}
\begin{description}
  \item[MO 467871] Geometric group theory thread guiding the symmetry discussion.
  \item[MO 466991] Descriptive set-theory spans motivating shape invariants.
\end{description}
\subsection*{Math.SE anchors}
\begin{description}
  \item[Math.SE 486912] "What is a cohesive topos?"
  \item[Math.SE 495120] "Operads detecting shape invariants".
\end{description}
\subsection*{Encyclopedia outline}
\begin{enumerate}
  \item \textbf{Cohesive background.} Review cohesive $\infty$-topoi, the shape modality, and the Math.SE 486912 primer so readers know why cohesion matters for operads.
  \item \textbf{Internal operads + spans.} Define operads internal to a cohesive topos, detail how spans encode operations, and highlight compatibility axioms that distinguish the cohesive setting from the condensed-span story in PR \#12.
  \item \textbf{Shape-invariance theorem.} Present the main comparison with classical Borsuk shape, pointing to MO 466991 for intuition about descriptive-set theoretic spans.
  \item \textbf{Condensed symmetry actions.} Explain how condensed groups act on cohesive operads, summarize MO 467871, and record the extra structure (shape/Galois mode) absent from prior proposals.
  \item \textbf{Examples + consequences.} Work through the condensed torus test case, discuss Galois actions on its homotopy fixed points, and list how these feed into future PlanetMath entries.
\end{enumerate}
\subsection*{Action items}
Sync arXiv source, pull MO embeddings (467871, 466991), pull Math.SE embeddings (486912, 495120), then expand the PlanetMath entry accordingly.
