% PlanetMath proposal draft for arXiv:2602.19155
% Source corpora: arXiv math.CT eprints, storage/mo-processed-gpu, storage/math-processed-gpu
\section*{Dualizing profinite spans in derivators}
\subsection*{Source}
\begin{itemize}
  \item arXiv:2602.19155, "Dualizing profinite spans in derivators" (2026-02-20)
  \item StackExchange cross-links via storage/mo-processed-gpu + math-processed-gpu
\end{itemize}
\subsection*{Synopsis}
This note promotes duality theorems for profinite spans in homotopy derivators. It explains how:
\begin{enumerate}
  \item profinite completion interacts with span composition,
  \item Verdier duality extends to derivator-valued spans, and
  \item compact generation is preserved under span-based Day convolution.
\end{enumerate}
Entry deliverables:
\begin{itemize}
  \item a glossary mapping "profinite span" and "derivator duality" to PlanetMath vocab,
  \item one detailed calculation for spans in $\mathsf{Pro(Fin)}$, and
  \item a diagram showing Day convolution vs. duality.
\end{itemize}
\subsection*{MathOverflow cues}
\begin{description}
  \item[MO 465967] Question on equivariant cohomology via Cech nerves. Provides context for duality statements in derivators.
  \item[MO 467551] Discussion of Green functions and harmonic analysis on groups where spans/duality appear.
\end{description}
\subsection*{Math.SE anchors}
\begin{description}
  \item[Math.SE 462715] "How does profinite completion commute with limits?" — helpful for the preliminaries.
  \item[Math.SE 467980] "Day convolution on profinite presheaves" — supports the final example.
\end{description}
\subsection*{Encyclopedia outline}
\begin{enumerate}
  \item Definitions: profinite spans, derivators, and Verdier duality.
  \item Statement of the main duality theorem and proof sketch.
  \item Interaction with Day convolution + compactness.
  \item Examples from $\mathsf{Pro(Fin)}$ and equivariant cohomology.
  \item References to the MO/Math.SE threads for extra context.
\end{enumerate}
\subsection*{Action items}
\begin{itemize}
  \item Sync the arXiv source into futon6 proof-state for citation harvest.
  \item Use storage/mo-processed-gpu to pull node/edge summaries for MO 465967 and MO 467551.
  \item Grab Math.SE embeddings for 462715 and 467980 via storage/math-processed-gpu.
  \item Draft the PlanetMath TeX + EDN pair following the outline.
\end{itemize}
