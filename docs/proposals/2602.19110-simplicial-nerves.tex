% PlanetMath proposal draft for arXiv:2602.19110
% Source corpora: arXiv math.CT eprints, storage/mo-processed-gpu, storage/math-processed-gpu
\section*{Simplicial nerves for internal bicategories}
\subsection*{Source}
\begin{itemize}
  \item arXiv:2602.19110, "Simplicial nerves for internal bicategories" (2026-02-20)
  \item StackExchange cross-links via storage/mo-processed-gpu + math-processed-gpu
\end{itemize}
\subsection*{Synopsis}
The paper develops a model for internal bicategories using simplicial nerves built inside a finitely complete ambient category. The key ingredients include:
\begin{enumerate}
  \item A comparison of Duskin-type nerves with Street nerves for internal bicategories,
  \item a rigidification procedure that uses codescent objects to pass from weak nerves to semistrict nerves, and
  \item an application to Grothendieck constructions for indexed bicategories.
\end{enumerate}
For the PlanetMath entry we will surface:
\begin{itemize}
  \item a dictionary converting the "simplicial refinement" jargon into pm-full-dictionary atoms (objects, spans, codescent, and horn fillers),
  \item a fully worked example in the bicategory of profunctors, and
  \item a note on how these nerves interact with base change in indexed bicategories.
\end{itemize}
\subsection*{MathOverflow cues}
\begin{description}
  \item[MO 467527] Question on higher-category terminology for semistrict nerves (tags: category-theory, higher-category-theory). We will cite this for definitions of "Gray nerves" versus "Duskin nerves".
  \item[MO 466863] Thread about D-modules and quiver varieties that references span-like nerves. We can map their intuition to the examples section.
\end{description}
\subsection*{Math.SE anchors}
\begin{description}
  \item[Math.SE 452731] "Explicit nerve of an internal groupoid" (ct.category-theory, reference-request) — good for an accessible example.
  \item[Math.SE 458240] "When do coskeletons detect equivalences?" (ct.category-theory, homotopy-theory) — supports the discussion of horn-filling conditions.
\end{description}
\subsection*{Encyclopedia outline}
\begin{enumerate}
  \item Definition of internal bicategory and recall of Duskin nerve.
  \item Construction of the semistrict nerve and theorem on equivalence with Street's construction.
  \item Codescent objects and their role in the rigidification.
  \item Examples: profunctors, indexed bicategories, and the Grothendieck construction.
  \item Bibliography + cross-links (MO 467527, MO 466863, Math.SE 452731, Math.SE 458240).
\end{enumerate}
\subsection*{Action items}
\begin{itemize}
  \item Extract the arXiv TeX into /home/joe/code/futon6/data/arxiv-math-ct-eprints/proof-state.
  \item Pull embeddings for MO 467527 and MO 466863 via storage/mo-processed-gpu hypergraph-thread-ids.
  \item Add Math.SE citations by querying storage/math-processed-gpu for thread IDs 452731 and 458240.
  \item Produce the PlanetMath EDN + TeX pair once outline is confirmed.
\end{itemize}
