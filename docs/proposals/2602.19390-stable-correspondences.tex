% PlanetMath proposal draft for arXiv:2602.19390
% Source corpora: arXiv math.CT eprints, storage/mo-processed-gpu, storage/math-processed-gpu
\section*{Stable correspondences for ambidextrous functors}
\subsection*{Source}
\begin{itemize}
  \item arXiv:2602.19390, "Stable correspondences for ambidextrous functors" (2026-02-21)
  \item StackExchange cross-links via storage/mo-processed-gpu + math-processed-gpu
\end{itemize}
\subsection*{Synopsis}
Stable correspondences encode ambidexterity (co/limit compatibilities) in symmetric monoidal $\infty$-categories. The paper:
\begin{enumerate}
  \item defines a bicategory $\mathrm{Corr}^{\mathrm{st}}(\mathcal{C})$ whose morphisms are spans together with stabilization data;
  \item proves an ambidexterity theorem showing that norms and transfers agree inside $\mathrm{Corr}^{\mathrm{st}}$;
  \item applies the theorem to motivic spectra and power operations; and
  \item provides explicit norm formulas explained via MathOverflow 466503/467551.
\end{enumerate}
\subsection*{MathOverflow cues}
\begin{description}
  \item[MO 466503] Number-theoretic spans giving intuition for norm/trace compatibilities.
  \item[MO 467551] Green-function spans used to motivate the spectral examples.
\end{description}
\subsection*{Math.SE anchors}
\begin{description}
  \item[Math.SE 492014] "Ambidexterity in stable homotopy".
  \item[Math.SE 494288] "Span correspondences and traces".
\end{description}
\subsection*{Encyclopedia outline}
\begin{enumerate}
  \item \textbf{Ambidexterity background.} Recall classical norm/trace identities, summarize Math.SE 492014, and explain why ambidextrous functors require both left and right adjoints inside symmetric monoidal $\infty$-categories.
  \item \textbf{Definition of $\mathrm{Corr}^{\mathrm{st}}$.} Construct the stabilization data inside the correspondence bicategory, spell out how spans gain transfers/norms, and contrast with the unstabilized setting referenced in MO 466503.
  \item \textbf{Ambidexterity theorem.} Outline the proof that norms and transfers coincide inside $\mathrm{Corr}^{\mathrm{st}}$, highlighting the mate calculus and pointing to the explicit norm formulae gathered from MO 467551.
  \item \textbf{Motivic spectrum case study.} Walk through the example for motivic spectra and power operations, noting how Math.SE 492014 supplies the input on ambidextrous functors.
  \item \textbf{Applications + future directions.} Describe the consequences for power operations (Math.SE 494288), explain how to detect ambidexterity in other settings, and list potential PlanetMath cross-links (spectral Mackey functors, trace methods).
\end{enumerate}
\subsection*{Action items}
Pull arXiv source, gather MO embeddings (466503, 467551), gather Math.SE embeddings (492014, 494288), draft the entry with the richer outline.
