% PlanetMath proposal draft for arXiv:2602.19364
% Source corpora: arXiv math.CT eprints, storage/mo-processed-gpu, storage/math-processed-gpu
\section*{Cyclotomic fibrations in parametrized higher categories}
\subsection*{Source}
\begin{itemize}
  \item arXiv:2602.19364, "Cyclotomic fibrations in parametrized higher categories" (2026-02-21)
  \item StackExchange cross-links via storage/mo-processed-gpu + math-processed-gpu
\end{itemize}
\subsection*{Synopsis}
A cyclotomic fibration is a locally constant family of $\infty$-categories over $S^1$ equipped with cyclotomic trace data. The authors:
\begin{enumerate}
  \item classify such fibrations via spans of monodromy automorphisms in $\mathrm{Aut}(\mathcal{C})$;
  \item show how the classification feeds topological cyclic homology (TC) calculations;
  \item express TC as a colimit over the span nerve, giving access to explicit polynomial models; and
  \item detail examples coming from polynomial endofunctors and cyclotomic spectra.
\end{enumerate}
\subsection*{MathOverflow cues}
\begin{description}
  \item[MO 465855] Dirichlet-series/cyclotomic discussion grounding the TC story.
  \item[MO 467935] Lie-theoretic spans highlighting how monodromy data behaves.
\end{description}
\subsection*{Math.SE anchors}
\begin{description}
  \item[Math.SE 486220] "Parametrized $\infty$-categories and monodromy".
  \item[Math.SE 488901] "What is a cyclotomic spectrum?".
\end{description}
\subsection*{Encyclopedia outline}
\begin{enumerate}
  \item Recall parametrized $\infty$-categories and the notion of monodromy via spans.
  \item Define cyclotomic fibrations and the span nerve used in the classification theorem.
  \item Explain the TC comparison, citing MO 465855 and Math.SE 488901.
  \item Work through a polynomial endofunctor example to show how the classification is computed.
  \item Summarize consequences for trace/norm operations.
\end{enumerate}
\subsection*{Action items}
Sync arXiv source, ingest MO embeddings (465855, 467935), ingest Math.SE embeddings (486220, 488901), and draft the expanded PlanetMath entry.
